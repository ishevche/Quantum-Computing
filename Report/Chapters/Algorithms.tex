% Chapter Template

\chapter{Considered algorithms} % Main chapter title

\label{ch:considered-algorithms} % Change X to a consecutive number; for referencing this chapter elsewhere, use \ref{ChapterX}

As a part of this project I have researched and implemented four basic algorithms:
\begin{itemize}
    \item Deutsch-Jozsa Algorithm
    \item Bernstein-Vazirani Algorithm
    \item Quantum Fourier Transform
    \item Quantum Phase Estimation
\end{itemize}

Also as a part of this project I have developed a laboratory work that aims to tech students basic operations using Qiskit development kit.

Let me explain researched algorithms in details.


\section{Deutsch-Jozsa Algorithm}

The Deutsch-Jozsa is a quantum algorithm proposed by David Deutsch and Richard Jozsa in 1992\cite{DeutschJozsa}.
Although of little current practical use, it is one of the simplest examples of a quantum algorithm that is exponentially faster than any possible classical algorithm, that makes it a good entry point.

\subsection{Problem statement}

In Deutsch-Jozsa problem we have a boolean function $f : \{0, 1\}^n \rightarrow \{0, 1\}$ that is known to be either balanced or constant.
The task is to determine whether the function $f$ is balanced or constant.

Constant function is the function that maps every $n$-bit input the same result (either 0 or 1).

Balanced function is the function that maps exactly a half of the inputs to 0 and a half to 1.

\subsection{Classic solution}

Solving this problem using classical algorithms requires $2^{n-1} + 1$ evaluations of the function $f$ for different inputs.
If among all received values there are different ones, then the function is balanced, and it is constant otherwise.

Worth noticing that problem is not always solvable using smaller amount of evaluations.
Really, in case of $2^{n-1}$ evaluations of function $f$ all received values could appear to be equal and in this case it is not possible to distinguish the following two cases:
\begin{itemize}
    \item the function is constant and for every input, output is the same
    \item the function is balanced and all evaluated inputs are from the same half
\end{itemize}

\subsection{Hadamard transform}\label{subsec:h_transform}

Before proceeding to quantum solution of Deutsch-Jozsa problem I would like to discuss Hadamard transformation first.

Let assume $n$ qubits in one of standard basis states.
In that case each qubit is in one of its basis state, either $\ket{0}$ or $\ket{1}$, so we can represent the whole system as ket vector with consequently written states of each individual qubit.
For example, system of two qubits where the first one is in state $\ket{1}$ and the second one is in state $\ket{0}$ can be written in the following way: $\ket{\psi} = \ket{1}_1 \otimes \ket{0}_2 = \ket{10}$

Furthermore notation writen inside ket vector can be considered as binary representation of some number, so we can represent our system in the following way: $\ket{\psi} = \ket{1}_1 \otimes \ket{0}_2 = \ket{10} = \ket{2}$

Let now apply Hadamard gate to every qubit in this system.
Applying this gate each qubit transforms in the following way:
\begin{align*}
    \ket{0} \rightarrow \frac{1}{\sqrt {2}}(\ket{0} + \ket{1}) \\
    \ket{1} \rightarrow \frac{1}{\sqrt {2}}(\ket{0} - \ket{1})
\end{align*}
so after applying Hadamard gates to system $\ket{2}$ it will be in the following state:
\begin{align*}
    H^{\otimes 2}\ket{2} &= H\ket{1}_1 \otimes H\ket{0}_2 \\
    &= \frac{1}{\sqrt {2}}(\ket{0}_1 - \ket{1}_1) \otimes \frac{1}{\sqrt {2}}(\ket{0}_2 + \ket{1}_2) \\
    &= \frac{1}{2} (\ket{00} + \ket{01} - \ket{10} - \ket{11})
\end{align*}

In this example we can see that state $\ket{2}$ transformed in superposition of all basis states with magnitude $\pm\frac{1}{2}$.
In fact, it is so for every basis state of $n$-qubit system with magnitude equal $\pm\frac{1}{\sqrt {2^n}}$ for general case.

The actual sign of the magnitude depends on a basis state $y$ and initial state $x$.
Each bit in state $y$ introduce $-$ sign if both this bit in $y$ equals $1$ and this bit in initial state $x$ also equals $1$, or simply if this bit in number $x \, AND \, y$ equals $1$, where $AND$ is bitwise AND operation.

Let's define operation $x \cdot y$ to be sum of bits of the number $x \, AND \, y$.
Then Hadamard transformation of the state $\ket{x}$ can be written in the following way:
\begin{align} \label{h_transformation}
    \ket{x} \rightarrow \frac{1}{\sqrt {2^n}} \sum_{y=0}^{2^n-1} (-1)^{x \cdot y} \ket{y}
\end{align}

\subsection{Quantum solution}\label{subsec:dj-quantum-solution}

The role of the function in quantum computer plays an oracle.
Oracle is a circuit that receives $n+1$ qubits as input in state $\ket{x}\otimes\ket{y}$ and transform it into state $\ket{x}\otimes\ket{y\oplus f(x)}$, where $\oplus$ is addition modulo 2.

Using the quantum properties we can receive the answer to the problem using oracle only once.
The whole algorithm can be described by following steps:
\begin{enumerate}
    \item Prepare the state.
    We have to prepare the state of $n+1$ qubits, first $n$ of which are initialized to $\ket{0}$ and the last one initialized to $\ket{1}$:
    \[\ket{\psi_0} = \ket{0^{\otimes n}}\otimes\ket{1}\]
    \item Apply Hadamard gate to each qubit.
    Using equation~\eqref{h_transformation} for the first $n$ qubits we can receive the following representation of transformed system:
    \begin{align*}
        \ket{\psi_1} &= \frac{1}{\sqrt{2^{n+1}}} \sum_{x=0}^{2^n-1}\left((-1)^{x\cdot0}\ket{x}\otimes\left(\ket{0} - \ket{1}\right)\right) \\
        &=\frac{1}{\sqrt{2^{n+1}}} \sum_{x=0}^{2^n-1}\left(\ket{x}\otimes\left(\ket{0} - \ket{1}\right)\right)
    \end{align*}
    \item Apply the oracle.
    After applying the state can be represented in the following way:
    \begin{align*}
        \ket{\psi_2} &= \frac{1}{\sqrt{2^{n+1}}} \sum_{x=0}^{2^n-1}\left(\ket{x}\otimes\left(\ket{0\oplus f(x)} - \ket{1\oplus f(x)}\right)\right) \\
        &=  \frac{1}{\sqrt{2^{n+1}}} \sum_{x=0}^{2^n-1}\left(\ket{x}\otimes(-1)^{f(x)}\left(\ket{0} - \ket{1}\right)\right)
    \end{align*}
    as $f(x)$ is either $1$ or $0$.
    \item Apply Hadamard gate to each of the first $n$ qubits.
    At this point we can discard the last qubit, and the state of first $n$ qubits before applying gates can be represented in the following way:
    \[
        \ket{\psi_2} = \frac{1}{\sqrt{2^{n}}} \sum_{x=0}^{2^n-1}(-1)^{f(x)}\ket{x}
    \]
    Applying equation~\eqref{h_transformation} we can represent the resulting state in the following form:
    \begin{align*}
        \ket{\psi_3} &= \frac{1}{2^{n}} \sum_{x=0}^{2^n-1}(-1)^{f(x)}\sum_{y=0}^{2^n-1}(-1)^{x\cdot y}\ket{y} \\
        &= \frac{1}{2^{n}} \sum_{y=0}^{2^n-1} \sum_{x=0}^{2^n-1}\left((-1)^{f(x)}(-1)^{x\cdot y}\right)\ket{y}
    \end{align*}
    Here $x\cdot y$ -- sum of bits of number $x \, AND \, y$ modulo 2, where $AND$ is bitwise AND operation.
    \item Perform measurement on the first $n$ qubits.
    Magnitude of state $\ket{0^{\otimes n}}$ in system is the following:
    \[
        \frac{1}{2^{n}}\sum_{x=0}^{2^n-1}\left((-1)^{f(x)}(-1)^{x\cdot 0}\right) = \frac{1}{2^{n}}\sum_{x=0}^{2^n-1}(-1)^{f(x)}
    \]
    so if function $f$ is constant $\sum(-1)^{f(x)}=\pm2^n$ and otherwise $\sum(-1)^{f(x)}=0$, as half of addends are $1$ and other half are $-1$.
    After measurement for constant function we will definitely receive $0$ as result, and for balanced function we will definitely not receive $0$.
\end{enumerate}

\section{Bernstein-Vazirani Algorithm}

The Bernstein-Vazirani is a quantum algorithm invented by Ethan Bernstein and Umesh Vazirani in 1997~\cite{BernsteinVazirani}.
This algorithm solves an extension of the Deutsch-Jozsa.

\subsection{Problem statement}

In Bernstein-Vazirani problem we have a boolean function $f:\{0, 1\}^n \rightarrow \{0, 1\}$ that is known to calculate dot product between input $x$ and secret string $s \in \{0, 1\}^n$ modulo 2.
The task is to determine the secret string.

\subsection{Classic solution}

Solving the problem using classical algorithms requires $n$ evaluations of the function with inputs $x = 2^i$ for $i=\overline{0,n-1}$.
The result of the evaluation at point $2^i$ is the $i'$s bit of the secret string:
\[
    f(2^i) = s_n\cdot0\oplus s_{n-1}\cdot0\oplus\ldots\oplus s_{i}\cdot1\oplus\ldots\oplus s_{0}\cdot0 = s_i
\]

Using this we can discover every bit of the secret string.

\subsection{Quantum solution}

Similarly to Deutsch-Jozsa solution~\eqref{subsec:dj-quantum-solution} the role of the function plays an oracle, that transforms state of $n+1$ qubits $\ket{x}\otimes\ket{y}$ into another state $\ket{x}\otimes\ket{y\oplus f(x)}$.

Bernstein-Vazirani problem is also can be solved using oracle only once.
The algorithm can be divided on the following steps:
\begin{enumerate}
    \item Prepare the state.
    Prepare the state of $n+1$ qubits, first $n$ of which are initialized to $\ket{0}$ and the last one initialized to $\ket{1}$:
    \[\ket{\psi_0} = \ket{0^{\otimes n}}\otimes\ket{1}\]
    \item Apply Hadamard gate to each qubit.
    Using equation~\eqref{h_transformation} for the first $n$ qubits we can receive the following representation of transformed system:
    \begin{align*}
        \ket{\psi_1} &= \frac{1}{\sqrt{2^{n+1}}} \sum_{x=0}^{2^n-1}\left((-1)^{x\cdot0}\ket{x}\otimes\left(\ket{0} - \ket{1}\right)\right) \\
        &=\frac{1}{\sqrt{2^{n+1}}} \sum_{x=0}^{2^n-1}\left(\ket{x}\otimes\left(\ket{0} - \ket{1}\right)\right)
    \end{align*}
    \item Apply the oracle.
    After applying the state can be represented in the following way:
    \begin{align*}
        \ket{\psi_2} &= \frac{1}{\sqrt{2^{n+1}}} \sum_{x=0}^{2^n-1}\left(\ket{x}\otimes\left(\ket{0\oplus f(x)} - \ket{1\oplus f(x)}\right)\right) \\
        &= \frac{1}{\sqrt{2^{n+1}}} \sum_{x=0}^{2^n-1}\left(\ket{x}\otimes(-1)^{f(x)}\left(\ket{0} - \ket{1}\right)\right) \\
        &= \frac{1}{\sqrt{2^{n+1}}} \sum_{x=0}^{2^n-1}\left((-1)^{s\cdot x}\ket{x}\otimes\left(\ket{0} - \ket{1}\right)\right)
    \end{align*}
    as $f(x)$ is either $1$ or $0$.
    \item Apply Hadamard gate to each of the first $n$ qubits.
    At this point we can discard the last qubit, and the state of first $n$ qubits before applying gates can be represented in the following way:
    \[
        \ket{\psi_2} = \frac{1}{\sqrt{2^{n}}} \sum_{x=0}^{2^n-1}(-1)^{s\cdot x}\ket{x}
    \]
    From the equation~\eqref{h_transformation} we know that applying Hadamard transformation to state $\ket{s}$ transforms it into state $\ket{\psi_2}$.
    As applying Hadamard gate to the qubit twice does not change the state of it, inverse operation for Hadamard transformation is the transformation itself, so applying Hadamard transformation on state $\ket{\psi_2}$ transforms it into state $\ket{\psi_3} = \ket{s}$.
    \item Perform measurement on the first $n$ qubits.
    As the system after previous step is in one of the basis states $\ket{s}$, the measurement will definitely possess the values $s$ -- unknown secret string.
\end{enumerate}

\section{Quantum Fourier Transform}

The Fourier transform is widely used throughout different algorithms on classical computers.
The quantum Fourier transform is just quantum implementation of the discrete Fourier transform.
The algorithm itself does not possess any interesting features, in fact, this algorithm simply transforms the state to a different basis.
Nevertheless, this algorithm is often used as a part of other algorith, most notably Shor's factoring and quantum phase estimation algorithms.

\subsection{Relation to discrete Fourier transform}

The discrete Fourier transform is a procedure that transforms an input vector $(x_1, x_2, \ldots x_N)$ into vector $(y_1, y_2, \ldots y_N)$ using the following rule:
\begin{align} \label{ft_rule}
    y_k = \frac{1}{\sqrt{N}}\sum_{j=0}^{N-1} x_j\cdot \omega^{jk}
\end{align}
where $\omega = e^{\frac{2\pi i}{N}}$.

The quantum Fourier transform transforms an input state $\ket{X} = \sum_{k=0}^{N-1} x_k\ket{k}$ into state $\ket{Y} = \sum_{k=0}^{N-1} y_k\ket{k}$ using the same rule as discrete Fourier transform~\eqref{ft_rule}

\subsection{Math behind quantum Fourier transform}

Let consider one of the basis states $\ket{X} = \ket{t}$ and derive how quantum Fourier transform will transform this state.
From the definition, the following holds:
\[
    y_k = \frac{1}{\sqrt{N}}\sum_{j=0}^{N-1}x_j\cdot\omega^{jk} = \frac{1}{\sqrt{N}}\cdot\omega^{tk}
\]
so, resulting state can be represented by following formula:
\[
    QFT\ket{t} = \frac{1}{\sqrt{N}}\sum_{j=0}^{N-1} \omega^{tj}\ket{j}
\]
Let define $n$ to be number of qubits in state $\ket{X}$, then $N=2^n$ as $N$ equals the number of basis states for this system.
Considering this we can rewrite new state in the following way:
\[
    QFT\ket{t} = \frac{1}{\sqrt{N}}\sum_{j=0}^{N-1}e^{2\pi i \frac{tj}{2^n}}\ket{j}
\]
We can represent $j$ in its binary form: $j = j_1 j_2 \ldots j_n$, where $j_1$ is most significant bit.
Then $j / 2^n = \sum_{k=1}^{n} j_k/2^k$, and the state can be rewritten:
\[
    QFT\ket{t} = \frac{1}{\sqrt{N}}\sum_{j=0}^{N-1}e^{2\pi i\left( \sum_{k=1}^{n} j_k/2^k \right)t}\ket{j}
\]
Now we can expand exponential of sum into product of exponential:
\[
    QFT\ket{t} = \frac{1}{\sqrt{N}} \sum_{j=0}^{N-1} \prod_{k=1}^{n} e^{2\pi i j_k t/2^k}\ket{j}
\]
And finally using similar logic as in section~\ref{subsec:h_transform} we can expand this sum of all basis states as tensor product across all individual qubits:
\begin{align} \label{eq_qft}
    QFT\ket{t} = \frac{1}{\sqrt{N}} \bigotimes_{k=1}^{n}\left( \ket{0} + e^{2\pi i t/2^k} \ket{1} \right)
\end{align}

\subsection{Circuit implementation}

To implement this algorithm we will use two gates: Hadamard gate ($H$) and controlled rotation gate ($CROT_k$).

From equation~\eqref{h_transformation} for $n=1$, we know that Hadamard gate performs the following transformation:
\[
    \ket{x} \mapsto \frac{1}{\sqrt {2}} \left( \ket{0} + (-1)^x \ket{1} \right)
\]
and using Euler's formula ($e^{\pi i} + 1 = 0$) we can rewrite it in the following way:
\[
    \ket{x} \mapsto \frac{1}{\sqrt {2}} \left( \ket{0} + \exp \left( \frac{2 \pi i}{2} x \right) \ket{1} \right)
\]

Unlike Hadamard's, $CROT_k$ is two-qubit gate.
The first of two qubits is control qubit, its value defines whether the rotation will be performed on the second qubit.
If control qubit is in state $\ket{1}$, then controlled bit transforms in the following way:
\begin{align*}
    \ket{0} &\mapsto \ket{0} \\
    \ket{1} &\mapsto \exp\left( \frac{2\pi i}{2^k} \right)\ket{1}
\end{align*}

This transformation is a very similar to Hadamard transformation, and can be written down in the similar way:
\[
    \ket{x} \mapsto \exp\left( \frac{2\pi i}{2^k} x \right)\ket{x}
\]
so applying gate $CROT_k$ to the two-qubit state results in the following state:
\begin{align*}
    CROT_k\ket{0x} &= \ket{0x} \\
    CROT_k\ket{1x} &= \exp\left( \frac{2\pi i}{2^k} x \right)\ket{1x}
\end{align*}
or simplifying it even more:
\[
    CROT_k\ket{xy} = \exp\left( \frac{2\pi i}{2^k} xy \right)\ket{xy}
\]

Given these two gates we can proceed to algorithm implementation.
Firstly, consider transformation of the first qubit:
\[
    \ket{t_1} \mapsto \frac{1}{\sqrt {2}}\left( \ket{0} + e^{ 2\pi i t/2^n }\ket{1} \right)
\]
As $t_1$ is the most significant bit in binary representation of $t$ it worth to write down $t$ using its bits:
\[
    t = 2^{n-1}t_1 + 2^{n-2}t_2 + \ldots + 2^{0}t_n
\]
Using this representation we can rewrite transformed bit in the following way:
\begin{align*}
    \ket{t_1} &\mapsto \frac{1}{\sqrt {2}}\left( \ket{0} + \exp \left( \frac{2\pi i}{2^n} \left( 2^{n-1}t_1 + 2^{n-2}t_2 + \ldots + 2^{0}t_n \right) \right)\ket{1} \right) \\
    \ket{t_1} &\mapsto \frac{1}{\sqrt {2}}\left( \ket{0} + \exp \left( \frac{2\pi i}{2} t_1 + \frac{2\pi i}{2^2}t_2 + \ldots + \frac{2\pi i}{2^n}t_n \right)\ket{1} \right)
\end{align*}

To achieve such transformation we can apply Hadamard gate to the first qubit and then consequently apply $CROT_i$ gates with qubit $i$ as control one for every $i$.

Using similar sequence of qubits $2 \dots n$ we can achieve the following state:
\begin{align*}
    \frac{1}{\sqrt {2}} \left[ \ket{0} + \exp \left( \frac{2\pi i}{2^n}x \right)\ket{1} \right] \otimes
    \frac{1}{\sqrt {2}} \left[ \ket{0} + \exp \left( \frac{2\pi i}{2^{n-1}}x \right)\ket{1} \right] \otimes \dots \\
    \dots \otimes
    \frac{1}{\sqrt {2}} \left[ \ket{0} + \exp \left( \frac{2\pi i}{2^2}x \right)\ket{1} \right] \otimes
    \frac{1}{\sqrt {2}} \left[ \ket{0} + \exp \left( \frac{2\pi i}{2^1}x \right)\ket{1} \right]
\end{align*}
which is almost the target state, the only differance is reversed order of qubits.


\section{Quantum Phase Estimation}

The quantum phase estimation is a quantum algorithm initially introduced by Alexei Kitaev in 1995~\cite{QPE}.
This algorithm acts as a main part for many other quantum algorithms.

\subsection{Problem statement}

Given a unitary matrix $U$ and a quantum state $\ket{\phi}$ that represents a eigenvector of matrix $U$.
Since $U$ is unitary, every eigenvalue has norm $1$, so the following holds: $U\ket{\phi} = e^{2\pi i \theta}\ket{\phi}$.
The task is to estimate the value of $\theta$

\subsection{Solution}

In quantum mechanics for every unitary matrix there is a corresponding operator.
Let consider operator $U$ that utilizes given unitary matrix.
Given this operator and state $\ket{\phi}$ the algorithm can be described by following steps:
\begin{enumerate}
    \item Prepare the state.
    We have to prepare the state of $n+1$ qubits, the last one initialized in state $\ket{\phi}$ and first $n$ of which are initialized in state $\ket{0}$.
    This qubits will be used to store value $2^n\theta$ on them, so the bigger n is, more precise result will be received.
    Initial state can be represented in a following way:
    \[
        \ket{\psi_0} = \ket{0}^{\otimes n}\otimes\ket{\phi}
    \]
    \item Apply Hadamard gate to each of the first $n$ qubits.
    This action will transform system in the following state:
    \[
        \ket{\psi_1} = \frac{1}{\sqrt {2^n}} \left( \ket{0} + \ket{1} \right)^{\otimes n}\otimes\ket{\phi}
    \]
    \item Apply unitary operations.
    We apply controlled version of operator $U$ $2^j$ times onto last bit controlled by bit $n-j$ for every $j = \overline{0, n-1}$.
    Remember that  $U$ is unitary operator with eigenvector $\ket{\phi}$, so $U\ket{\phi} = e^{2\pi i \theta}\ket{\phi}$ and as a result:
    \[
        U^2^j\ket{\phi} = U^{2^j-1}e^{2\pi i \theta}\ket{\phi} = \dots = e^{2\pi i 2^j\theta}\ket{\phi}
    \]
    Applying controlled operator for control bit $n-j$ yields the following equation:
    \[
        CU^2^j\left( \left( \ket{0} + \ket{1} \right) \otimes \ket{\phi} \right) = \left( \ket{0} + e^{2\pi i 2^j\theta}\ket{1} \right) \otimes \ket{\phi}
    \]
    And after applying all operators the state of the system will be the following:
    \[
        \ket{\psi_2} = \frac{1}{\sqrt {2^n}} \left( \ket{0} + e^{2\pi i 2^{n-1}\theta}\ket{1} \right) \otimes \dots \otimes \left( \ket{0} + e^{2\pi i 2^0\theta}\ket{1} \right) \otimes \ket{\phi}
    \]
    \item Apply inverse Fourier transform.
    If we substitute $x = 2^n\theta$ in the equation~\eqref{eq_qft} we will receive the state of the first $n$ qubits in $\ket{\psi_2}$, so current state can be described in the following way:
    \[
        \ket{\psi_2} = ()QFT\ket{2^n\theta}) \otimes \ket{\phi}
    \]
    so after applying inverse quantum Fourier transform to first n qubits of the system, we will receive the following state:
    \[
        \ket{\psi_3} = \ket{2^n\theta} \otimes \ket{\phi}
    \]
    \item Perform measurement on the first $n$ qubits.
    As the system after previous step is in one of the basis states $\ket{2^n\theta}$, the measurement will definitely possess the value $2^n\theta$.
\end{enumerate}


