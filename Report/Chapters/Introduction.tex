% Chapter 1

\chapter{Introduction} % Main chapter title

\label{Introduction} % For referencing the chapter elsewhere, use \ref{Chapter1} 

%----------------------------------------------------------------------------------------

% Define some commands to keep the formatting separated from the content 
\newcommand{\keyword}[1]{\textbf{#1}}
\newcommand{\tabhead}[1]{\textbf{#1}}
\newcommand{\code}[1]{\texttt{#1}}
\newcommand{\file}[1]{\texttt{\bfseries#1}}
\newcommand{\option}[1]{\texttt{\itshape#1}}

%----------------------------------------------------------------------------------------

Quantum computing is a field of computer science that has recently emerged at the forefront of the scientific world.
Exploiting the principles of quantum mechanics, quantum computers possess capabilities never seen before.
Such a wide range of available opportunities holds immense promise for finding new algorithms to solve complex problems that are beyond the reach of classical computers now.
In this project, I delve into the realm of quantum computing and explore its potential applications and implications for the future.

Because of the constraints of physics, the rapid growth of computing power described by Moore's Law can no longer maintain the same rate.
However, demand for computational power still rises exponentially.
This is where quantum computing could enter the scene, offering solutions to overcome the boundaries imposed by classical computers.

At the underlying layer of quantum computers lie the principles of quantum mechanics, and by leveraging its properties, quantum computers can perform computations in ways classical computers cannot.
This, in turn, provides the potential to solve certain problems significantly better.

Within this project, I have extensively researched and analyzed four algorithms that significantly benefit from the properties of quantum mechanics.
Already, these algorithms demonstrate the huge potential that quantum computing possesses.

After understanding the underlying principles of these algorithms, I have also developed a laboratory component that aims to introduce students to the fascinating world of quantum computing.

Through this project, I aim to shed light on the immense potential and significance of quantum computing in the modern world.
By delving into the algorithms that benefit from the properties of quantum mechanics and providing a practical laboratory experience, I hope to contribute to the understanding of the insights of this fascinating field of study.


%----------------------------------------------------------------------------------------


